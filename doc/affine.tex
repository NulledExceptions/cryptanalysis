\section{Аффинный шифр}

Аффинный шифр — это частный случай более общего моноалфавитного шифра 
подстановки. К шифрам подстановки относятся также шифр Цезаря, 
ROT13 и Атбаш. Поскольку аффинный шифр легко дешифровать, 
он обладает слабыми криптографическими свойствами.


\subsection{Описание}

В аффинном шифре каждой букве алфавита размера m ставится в 
соответствие число из диапазона $[0, .. ,m-1]$. 
Затем при помощи модульной арифметики для каждого числа, соответствующего 
букве исходного алфавита, вычисляется новое число, которое заменит 
старое в шифротексте. Функция шифрования для каждой буквы

    $\mbox{E}(x)=(ax+b)\mod{m}$,

где модуль m — размер алфавита, а пара a и b — ключ шифра. Значение 
a должно быть выбрано таким, что a и m — взаимно простые числа. Функция 
расшифрования

    $\mbox{D}(x)=a^{-1}(x-b)\mod{m}$,

где $a^{-1}$ — обратное к a число по модулю m. То есть оно удовлетворяет 
уравнению

    $1 = a a^{-1}\mod{m}$.

Обратное к a число существует только в том случае, когда a и m — взаимно 
простые. Значит, при отсутствии ограничений на выбор числа a расшифрование 
может оказаться невозможным. Покажем, что функция расшифрования является 
обратной к функции шифрования


    $\begin{matrix}\mbox{D}(\mbox{E}(x)) &= &a^{-1}(\mbox{E}(x)-b)\mod{m}\\ &= &a^{-1}(((ax+b)\mod{m})-b)\mod{m} \\ &= &a^{-1}(ax+b-b)\mod{m} \\ &= &a^{-1}ax \mod{m}\\ & = &x\mod{m}. \end{matrix}$

Количество возможных ключей для аффинного шифра можно записать 
через функцию Эйлера как $\varphi(m)*m$.


\subsection{Примеры шифрования и расшифрования}

В следующих примерах используются латинские буквы 
от A до Z, соответствующие им численные значения приведены в таблице.

A 	B 	C 	D 	E 	F 	G 	H 	I 	J 	K 	L 	M 	N 	O 	P 	Q 	R 	S 	T 	U 	V 	W 	X 	Y 	Z
0 	1 	2 	3 	4 	5 	6 	7 	8 	9 	10 	11 	12 	13 	14 	15 	16 	17 	18 	19 	20 	21 	22 	23 	24 	25


\subsection{Шифрование}

В этом примере необходимо зашифровать сообщение "ATTACK AT DAWN
", используя упомянутое выше соответствие между буквами и числами, 
и значения a=3, b=4 и m=26, так как в используемом алфавите 26 букв. 
Только на число a наложены ограничения, так как оно должно быть взаимно 
простым с 26. Возможные значения a: 1, 3, 5, 7, 9, 11, 15, 17, 19, 
21, 23 и 25. Значение b может быть любым, только если a не равно единице
, так как это сдвиг шифра. Итак, для нашего примера функция шифрования 
$y=E(x)=(3x+4)\pmod{26}$. Первый шаг шифрования — запись чисел, 
соответствующих каждой букве сообщения.

сообщение 	A 	T 	T 	А 	C 	K 	A 	T 	D 	A 	W 	N
x 	0 	19 	19 	0 	2 	10 	0 	19 	3 	0 	22 	13

Теперь, для каждого значения x найдем значение (3x+4). После нахождения 
значения (3x+4) для каждого символа возьмем остаток от деления (3x+4) 
на 26. Следующая таблица показывает первые четыре шага процесса 
шифрования.

сообщение 	A 	T 	T 	А 	C 	K 	A 	T 	D 	A 	W 	N
x 	0 	19 	19 	0 	2 	10 	0 	19 	3 	0 	22 	13
$3x+4$ 	4 	61 	61 	4 	10 	34 	4 	61 	13 	4 	70 	43
$(3x+4)\pmod{26}$ 	4 	9 	9 	4 	10 	8 	4 	9 	13 	4 	18 	17

Последний шаг процесса шифрования заключается в подстановке вместо каждого 
числа соответствующей ему буквы. В этом примере шифротекст будет 
"EJJEKIEJNESR". Таблица ниже показывает все шаги по шифрованию сообщения 
аффинным шифром.

сообщение 	A 	T 	T 	А 	C 	K 	A 	T 	D 	A 	W 	N
x 	0 	19 	19 	0 	2 	10 	0 	19 	3 	0 	22 	13
$3x+4$ 	4 	61 	61 	4 	10 	34 	4 	61 	13 	4 	70 	43
$(3x+4)\pmod{26}$ 	4 	9 	9 	4 	10 	8 	4 	9 	13 	4 	18 	17
шифротекст 	E 	J 	J 	E 	K 	I 	E 	J 	N 	E 	S 	R


\subsection{Расшифрование}

Для расшифрования возьмем шифротекст из примера с шифрованием. Функция 
расшифрования будет $\mbox{D}(y)=a^{-1}(y-b)\mbox{ mod }m$, где $a^{-1}=9$, 
$b=4$ и $m=26$. Для начала запишем численные значения для каждой буквы 
шифротекста, как показано в таблице ниже.

шифротекст 	E 	J 	J 	E 	K 	I 	E 	J 	N 	E 	S 	R
y 	4 	9 	9 	4 	10 	8 	4 	9 	13 	4 	18 	17

Теперь рассчитаем для каждого y необходимо рассчитать 9(y-4) и взять 
остаток от деления этого числа на 26. Следующая таблица показывает 
результат этих вычислений.

шифротекст 	E 	J 	J 	E 	K 	I 	E 	J 	N 	E 	S 	R
y 	4 	9 	9 	4 	10 	8 	4 	9 	13 	4 	18 	17
$9(y-4)$ 	0 	45 	45 	0 	54 	36 	0 	45 	81 	0 	126 	117
$9(y-4)\pmod{26}$ 	0 	19 	19 	0 	2 	10 	0 	19 	3 	0 	22 	13

Последний шаг операции расшифрования для шифротескста — поставить в 
соответствие числам буквы. Сообщение после расшифрования будет 
"ATTACKATDAWN". Таблица ниже показывает выполнение последнего шага.

шифротекст 	E 	J 	J 	E 	K 	I 	E 	J 	N 	E 	S 	R
y 	4 	9 	9 	4 	10 	8 	4 	9 	13 	4 	18 	17
9(y-4) 	0 	45 	45 	0 	54 	36 	0 	45 	81 	0 	126 	117
$9(y-4)\pmod{26}$ 	0 	19 	19 	0 	2 	10 	0 	19 	3 	0 	22 	13
сообщение 	A 	T 	T 	А 	C 	K 	A 	T 	D 	A 	W 	N


\subsection{Шифрование всего алфавита}

Чтобы ускорить шифрование и расшифрование, можно провести процедуру 
шифрования для всех букв алфавита и получить таблицу соответствий 
между буквами исходного сообщения и шифротекста. Для использованных 
выше примеров такая таблица будет выглядеть следующим образом:

буква сообщения 	A 	B 	C 	D 	E 	F 	G 	H 	I 	J 	K 	L 	M 	N 	O 	P 	Q 	R 	S 	T 	U 	V 	W 	X 	Y 	Z
x 	0 	1 	2 	3 	4 	5 	6 	7 	8 	9 	10 	11 	12 	13 	14 	15 	16 	17 	18 	19 	20 	21 	22 	23 	24 	25
$(3x+4)\pmod{26}$ 	4 	7 	10 	13 	16 	19 	22 	25 	2 	5 	8 	11 	14 	17 	20 	23 	0 	3 	6 	9 	12 	15 	18 	21 	24 	1
буква шифротекста 	E 	H 	K 	N 	Q 	T 	W 	Z 	C 	F 	I 	L 	O 	R 	U 	X 	A 	D 	G 	J 	M 	P 	S 	V 	Y 	B


\subsection{Криптоанализ}

Так как аффинный шифр является по сути моноалфавитным шифром замены
, то он обладает всеми уязвимостями этого класса шифров. Шифр Цезаря 
— это аффинный шифр с a=1, что сводит функцию шифрования к простому 
линейному сдвигу.

В случае шифрования сообщений на русском языке (т.е. m=33) существует 
627 нетривиальных аффинных шифров, не учитывая 33 тривиальных шифра 
Цезаря. Это число легко посчитать, зная, что существует всего 20 чисел 
взаимно простых с 33 и меньших 33 (а это и есть возможные значения 
a). Каждому значению a могут соответствовать 33 разных дополнительных 
сдвига (значение b); то есть всего существует 2033 или 660 возможных 
ключей. Аналогично, для сообщений на английском языке (т.е. m=26) 
всего существует 1226 или 312 возможных ключей. Такое ограниченное 
количество ключей приводит к тому, что система крайне не криптостойка 
с точки зрения принципа Керкгоффса.

Основная уязвимость шифра заключается в том, что криптоаналитик может 
выяснить (путем частотного анализа, полного перебора, угадывания или 
каким-либо другим способом) соответствие между двумя любыми буквами 
исходного текста и шифротекста. Тогда ключ может быть найдет путем 
решения системы уравнений. Кроме того, так мы знаем, что a и m — взаимно 
простые, это позволяет уменьшить количество проверяемых ключей для 
полного перебора.

Преобразование, подобное аффинному шифру, используется в линейном 
конгруэнтном методе (разновидности генератора псевдослучайных чисел
). Этот метод не является криптостойким по той же причине, что и аффинный 
шифр.

