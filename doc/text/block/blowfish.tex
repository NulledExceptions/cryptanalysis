\subsection{Blowfish}
Blowfish - это алгоритм, разработанный Брюсом Шнайером специально для реализации на больших микропроцессорах. Алгоритм Blowfish не запатентован.

Алгоритм Blowfish оптимизирован для применения в системах, не практикующих частой смены ключей, например, в линиях связи и программах автоматического шифрования файлов. При реализации на 32-битовых микропроцессорах с большим размером кэша данных, например, процессорах Pentium и PowerPC, алгоритм Blowfish заметно быстрее DES. Алгоритм Blowfish не годится для применения в случаях, где требуется частая смена ключей, например, в коммутаторах пакетов, или в качестве однонаправленной хэш-функции. Большие требования к памяти не позволяют использовать этот алгоритм в смарт-картах.

\subsubsection{Описание}

Blowfish представляет собой 64-битовый блочный алгоритм шифрования с ключом переменной длины. Алгоритм состоит из двух частей: расширения ключа и шифрования данных. Расширение ключа преобразует ключ длиной до 448 битов в несколько массивов подключей общим размером 4168 байт.
Шифрование данных заключается в последовательном исполнении простой функции 16 раз. На каждом раунде выполняются зависимая от ключа перестановка и зависимая от ключа и данных подстановка. Используются только операции сложения и XOR над 32-битовыми словами. Единственные дополнительные операции каждого раунда - четыре взятия данных из индексированного массива.
В алгоритме Blowfish используется множество подключей. Эти подключи должны быть вычислены до начала зашифрования или расшифрования данных.

Каждый из четырех 32-битовых S-блоков содержит 256 элементов: 

$$S1,0, S1,1,…, S1,255$$
$$S2,0, S2,2,…, S2,255$$
$$S3,0, S3,3,…, S3,255$$
$$S4,0, S4,4,…, S4,255$$

Алгоритм Blowfish представляет собой сеть Файстеля, состоящей из 16 раундов. На вход подается 64-битовый элемент данных х. Для зашифрования данных:
Разбить  х на  две  32-битовых половины: xL, xR
Для i от 1 до 16:
		$$xL = xL    Pi$$
		$$xR = F (xL)    xR$$
	Переставить xL и xR (отнять последнюю перестановку)
$$xR = xR    P17$$
$$xL = xL    P18$$

	Объединить xL и xR
	
	Разделить xL на четыре 8-битовых фрагмента: а, b, с и d 

	$$F(xL) = ((S1,a + S2,b\mod232)   S3,c) + S4,d\mod232                                           $$
Расшифрование выполняется точно так же, как и зашифрование,  но Р1,Р2,...,Р18 используются в обратном порядке.

	В реализациях Blowfish, в которых требуется очень высокая скорость, цикл должен быть развернут, а все ключи храниться в кэше.
	Подключи рассчитываются с помощью самого алгоритма Blowfish. Вот какова точная последовательность действий.

1. Сначала Р-массив, а затем четыре S-блока по порядку инициализируются фиксированной строкой. Эта строка состоит из шестнадцатеричных цифр π.

2. Выполняется операция XOR над Р1 с первыми 32 битами ключа, XOR над Р2 со вторыми 32 битами ключа, и т.д. для всех битов ключа (вплоть до Р18). Операция XOR выполняется циклически над битами ключа до тех пор, пока весь Р-массив не будет инициализирован.

3. Используя подключи, полученные на этапах 1 и 2, алгоритм Blowfish шифрует строку из одних нулей.

4. Р1 и Р2 заменяются результатом этапа 3.

5. Результат этапа 3 шифруется с помощью алгоритма Blowfish и модифицированных подключей.

6. Р3 и Р4 заменяются результатом этапа 5.

7. Далее по ходу процесса все элементы Р-массива, а затем все четыре S-блока по порядку заменяются выходом постоянно меняющегося алгоритма Blowfish.

	Всего для генерации всех необходимых подключей требуется 521 итерация. Приложения могут сохранять подключи - нет необходимости выполнять процесс их получения многократно.

\subsubsection{Стойкость}

Серж Воденэ (Serge Vaudenay) исследовал алгоритм Blowfish с известными S-блоками и r раундами. Как оказалось, дифференциальный криптоанализ может восстановить Р-массив с помощью 28r+1 подобранных открытых текстов. Для некоторых слабых ключей, которые генерируют плохие S-блоки (вероятность выбора такого ключа составляет 1/214), эта же атака восстанавливает Р-массив с помощью всего 24г+1 подобранных открытых текстов. При неизвестных S-блоках эта атака может обнаружить использование слабого ключа, но не может восстановить сам ключ (и также S-блоки и Р-массив). Эта атака эффективна только против вариантов с уменьшенным числом раундов и совершенно безнадежна против 16-раундового алгоритма Blowfish. Разумеется, важно и открытие слабых ключей, хотя они, вероятно, использоваться не будут. Слабым называют ключ, для которого два элемента данного S-блока идентичны. До выполнения расширения ключа невозможно установить факт слабости ключа. 

	Не известны факты успешного криптоанализа алгоритма Blowfish. В целях безопасности не следует реализовывать Blowfish с уменьшенным числом раундов. Компания Kent Marsh Ltd. встроила алгоритм Blowfish в свой продукт FolderBolt, предназначенный для обеспечения защиты Microsoft Windows и Macintosh. Кроме того, алгоритм входит в Nautilus и PGPfone.
