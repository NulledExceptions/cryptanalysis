\DEF\textit{Лингвистика} (от лат. lingua — язык) — наука, это наука о 
всех языках мира как индивидуальных его представителях. В широком смысле 
слова, лингвистика подразделяется на научную и практическую.

\DEF\textit{Компьютерная лингвистика} — научное направление в области математического 
и компьютерного моделирования интеллектуальных процессов у человека и животных 
при создании систем искусственного интеллекта, которое ставит своей целью 
использование математических моделей для описания естественных языков.

\DEF\textit{Анализ текста} — процесс получения высококачественной информации из текста 
на естественном языке. Как правило, для этого применяется статистическое 
обучение на основе шаблонов: входной текст разделяется с помощью шаблонов,
затем производится обработка полученных данных.

Первоначально методы криптоанализа основывались на лингвистических закономерностях 
естественного текста и реализовывались с использованием только карандаша 
и бумаги. Со временем в криптоанализе нарастает роль чисто математических 
методов, для реализации которых используются специализированные криптоаналитические 
компьютеры.

