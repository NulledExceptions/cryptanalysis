\DEF\textit{Лингвистика} — наука, это наука о 
всех языках мира как индивидуальных его представителях. В широком смысле 
слова, лингвистика подразделяется на научную и практическую.

Первоначально методы криптоанализа основывались на лингвистических закономерностях 
естественного текста и реализовывались с использованием только карандаша 
и бумаги. Со временем в криптоанализе нарастает роль чисто математических 
методов, для реализации которых используются специализированные криптоаналитические 
компьютеры.

\DEF\textit{Компьютерная лингвистика} — научное направление в области математического 
и компьютерного моделирования интеллектуальных процессов у человека и животных 
при создании систем искусственного интеллекта, которое ставит своей целью 
использование математических моделей для описания естественных языков.

Современная лингвистика обладает мощными методами анализа языковых структур, 
в том числе синтеза и анализа. В этой работе внимание концентрируется только на 
последних.

\DEF\textit{Анализ текста} — процесс получения высококачественной информации из текста 
на естественном языке. Как правило, для этого применяется статистическое 
обучение на основе шаблонов: входной текст разделяется с помощью шаблонов,
затем производится обработка полученных данных.

Возможен анализ документа, написанного на неизвестном языке и/или неизвестной 
системой письма, но это так-же выходит за рамки данной работы. Разобран вопрос 
возможности автоматического определения корректности текста по самому 
тексту и языку его написания. В целях данного исследования реализованы простейшие
методы подобного анализа и проведено сравнение их корректности и скорости работы.

Тестирование каждого метода - запуск с романом «Война и мир» Льва Николаевича 
Толстого в качестве входных данных. Такой выбор текста обусловлен его 
легендарной длинной, что даст корректное представление о эффективности 
метода по памяти и по времени, и вкраплением в текст иностранных слов 
и терминов, что покажет общую корректность метода.

