Решение проблемы поиска открытого текста по шифровке всегда возможно 
свести к некоторой математической задаче. В данной работе
изучается только текстовые формы шифров, поэтому мостом 
для перевода изначальной проблемы в математическую задачу
будет служить лингвистика.

\DEF\textit{Лингвистика} — наука, это наука о 
всех языках мира как индивидуальных его представителях. В широком смысле 
слова, лингвистика подразделяется на научную и практическую.

Первоначально все методы криптоанализа основывались на 
лингвистических закономерностях 
естественного текста и реализовывались с использованием только карандаша 
и бумаги. Со временем в криптоанализе нарастает роль чисто математических 
методов, и такие методы уже сформировали свой раздел в 
лингвистике.

\DEF\textit{Компьютерная лингвистика} — научное направление в области 
математического и компьютерного моделирования интеллектуальных 
процессов у человека и животных 
при создании систем искусственного интеллекта, которое ставит своей целью 
использование математических моделей для описания естественных языков.

Современная лингвистика обладает мощными методами анализа языковых 
структур, в том числе методы синтеза и анализа. В этой работе 
внимание уделено только последним.

\DEF\textit{Анализ текста} — процесс получения информации из текста 
на естественном языке. Как правило, для этого применяется статистическое 
обучение на основе шаблонов: входной текст разделяется с помощью шаблонов,
затем производится обработка полученных данных.

Возможен анализ документа, написанного на неизвестном языке и/или 
неизвестной системой письма, но это так-же выходит за рамки данной работы. 

Во время анализа шифротекста бывает полезно попробовать расшифровать 
текст на каком-то подмножестве ключей и посмотреть результаты.
На основе того, что какой-то текст выглядит более или менее 
похоже на русский (или любой другой рассматриваемый язык), 
мы можем заключить что ключ более или менее хорош.
Итак, можно вывести две интересующие нас проблемы:

\begin{enumerate}
\item возможность определения языка текста по шифротексту;
\item возможность определения корректности текста по самому тексту 
и языку его написания (метрика корректности).
\end{enumerate}

Проблема 1) выглядит неразрешимо, она должна решаться для
каждого шифротекста отдельно — тогда возможно стороить гипотезы 
на основе характеристик оппонента. В данной работе выполнен только 
простейший анализ шифротекста на наличие лигатур и 
диактрических знаков характерных для языка.

\DEF\textit{Лигатура} — знак любой системы письма или 
фонетической транскрипции, образованный путем соединения 
двух и более знаков.

\DEF\textit{Диактрические знаки} — различные надстрочные, подстрочные,
реже внутристрочные знаки, применяемые в буквенных (в том числе 
консонантных) и слоговых системах письма не как самостоятельные 
обозначения звуков, а для изменения или уточнения значения других 
знаков.

Каждый язык для фреймворка выглядит как словарь ($dict$ в нотации
Python). Для примера, французский язык:

\begin{verbatim}
    'fr' :
        { 'A':8.11, 'À':8.11, 'Â':8.11, 'Æ':8.11, 
          'B':0.91, 'C':3.49, 'Ç':3.49, 'D':4.27, 
          'E':17.22, 'É':17.22, 'È':17.22, 'Ê':17.22, 
          'Ë':17.22, 'F':1.14, 'G':1.09, 'H':0.77, 
          'I':7.44, 'Î':7.44, 'Ï':7.44, 'J':0.34, 
          'K':0.09, 'L':5.53, 'M':2.89, 'N':7.46, 
          'O':5.38, 'Ô':5.38, 'Œ':5.38, 'P':3.02, 
          'Q':0.99, 'R':7.05, 'S':8.04, 'T':6.99, 
          'U':5.65, 'Ù':5.65, 'Û':5.65, 'Ü':5.65, 
          'V':1.30, 'W':0.04, 'X':0.44, 'Y':0.27, 
          'Ÿ':0.27, 'Z':0.09, 
          'max':17.22, 
          'kappa':0.0746 },
\end{verbatim}

Первый ключ всегда является кодом языка по стандарту ISO 639-1:2002.
Далее идет перечень всех букв с частотой их встречаемости и 
некоторые характеристики языка. Причем некоторые буквы имеют
несколько вариантов написания (например U с четырю вариантами).
Процент встречаемости встречаемости таких вариантов
и служит меткой языка.

Проблема 2) более прозаична так как имеет несколько методов 
решения, необходимо только выбрать наиболее подходящий.
В целях данного 
исследования реализованы простейшие методы подобного анализа и 
проведено сравнение их корректности и скорости работы.

Тестирование каждого метода - запуск с романом «Война и мир» Льва 
Николаевича 
Толстого в качестве входных данных. Такой выбор текста обусловлен его 
легендарной длинной, что даст корректное представление о эффективности 
метода по памяти и по времени, и вкраплением в текст иностранных слов 
и терминов, что покажет общую корректность метода.
