В работе \cite{shannon-inf} Клод Шеннон обобщил накопленный до него опыт 
разработки шифров. Оказалось, что даже в сложных шифрах в качестве 
типичных компонентов можно выделить шифры замены, шифры перестановки 
или их сочетания.

\DEF\textit{Шифры перестановки} — такие шифры, преобразования из которых 
приводят к изменению только порядка следования символов исходного 
сообщения. 

Обычно открытый текст разбивается на отрезки равной длины 
и каждый отрезок шифруется независимо. Пусть, например, длина отрезков 
равна $n$ и $\sigma$ — взаимно-однозначное отображение множества 
${1, 2, \dots, n}$ в себя.
Тогда шифр перестановки действует так: отрезок открытого 
текста $x_1, \dots, x_n$ преобразуется в отрезок шифрованного текста 
$x \sigma(1)\dots x \sigma(n)$. 

Простейший шифр перестановки — шифр Сциталь.

\DEF\textit{Шифры замены} — такие шифры, преобразования из которых приводят 
к замене каждого символа открытого сообщения на другие символы - шифробозначения
, причем порядок следования шифробозначений совпадает с порядком следования 
соответствующих им символов открытого сообщения. 

Дадим математическое описание шифра замены. 
Пусть $X$ и $Y$ — два алфавита открытого и соответственно 
шифрованного текстов, состоящие из одинакового числа символов. Пусть 
также $g: X \rightarrow Y$ - взаимнооднозначное отображение $X$ в $Y$. Это значит,
что каждой букве x алфавита X соответствует однозначно определенная 
буква y алфавита $Y$, которую мы обозначаем символом $g(x)$, причем 
разным буквам соответствуют разные. Тогда шифр замены действует так:
открытый текст $x_1, x_2, \dots, x_n$ преобразуется в шифрованный текст 
$g(x1), g(x2), \dots g(x_n)$. 

Простейший шифр замены — шифр Цезаря. 
