\subsection{Энигма}
Эни́гма (от греч. αίνιγμα — загадка) — портативная шифровальная машина, использовавшаяся для шифрования и дешифрования секретных сообщений. Более точно, Энигма — целое семейство электромеханических роторных машин, применявшихся с 20-х годов XX века.

Энигма использовалась в коммерческих целях, а также в военных и государственных службах во многих странах мира, но наибольшее распространение получила в нацистской Германии во время Второй мировой войны — именно Энигма вермахта (Wehrmacht Enigma) — германская военная модель — чаще всего является предметом дискуссий.

\subsubsection{Описание}
Как и другие роторные машины, Энигма состояла из комбинации механических и электрических подсистем. Механическая часть включала в себя клавиатуру, набор вращающихся дисков — роторов, — которые были расположены вдоль вала и прилегали к нему, и ступенчатого механизма, двигающего один или несколько роторов при каждом нажатии на клавишу.

Конкретный механизм работы мог быть разным, но общий принцип был таков: при каждом нажатии на клавишу самый правый ротор сдвигается на одну позицию, а при определённых условиях сдвигаются и другие роторы. Движение роторов приводит к различным криптографическим преобразованиям при каждом следующем нажатии на клавишу на клавиатуре.

Механические части двигались, замыкая контакты и образуя меняющийся электрический контур (то есть, фактически, сам процесс шифрования букв реализовывался электрически). При нажатии на клавишу клавиатуры контур замыкался, ток проходил через различные цепи и в результате включал одну из набора лампочек, и отображавшую искомую букву кода. (Например: при шифровке сообщения, начинающегося с ANX…, оператор вначале нажимал на клавишу A — загоралась лампочка Z — то есть Z и становилась первой буквой криптограммы. Далее оператор нажимал N и продолжал шифрование таким же образом далее).

Таким образом, постоянное изменение электрической цепи, через которую шёл ток, вследствие вращения роторов позволяло реализовать многоалфавитный шифр подстановки, что давало высокую, для того времени, устойчивость шифра.

Роторы — сердце Энигмы. Каждый ротор представлял собой диск примерно 10 см в диаметре, сделанный из эбонита или бакелита, с пружинными штыревыми контактами на одной стороне ротора, расположенными по окружности. На другой стороне находилось соответствующее количество плоских электрических контактов. Штыревые и плоские контакты соответствовали буквам в алфавите (обычно это были 26 букв от A до Z). При соприкосновении контакты соседних роторов замыкали электрическую цепь. Внутри ротора каждый штыревой контакт был соединён с одним из плоских. Порядок соединения мог быть различным.
Три ротора и шпиндель, к которому они крепятся.

Сам по себе ротор производил очень простой тип шифрования: элементарный шифр замены. Например, контакт, отвечающий за букву E, мог быть соединён с контактом буквы T на другой стороне ротора. Но при использовании нескольких роторов в связке (обычно трёх или четырёх) за счёт их постоянного движения получается более надёжный шифр.

Преобразование Энигмы для каждой буквы может быть определено математически как результат перестановок. Рассмотрим трёхроторную армейскую модель. Положим, что P обозначает коммутационную панель, U обозначает отражатель, а L, M, R обозначают действия левых, средних и правых роторов соответственно. Тогда шифрование E может быть выражено как:

    $$E = PRMLUL^{-1}M^{-1}R^{-1}P^{-1}$$

После каждого нажатия клавиш ротор движется, изменяя трансформацию. Например, если правый ротор R проворачивается на i позиций, происходит трансформация $\rho^iR\rho^{-i}$, где ρ — циклическая перестановка, проходящая от A к B, от B к C, и так далее. Таким же образом, средний и левый ротор могут быть обозначены как j и k вращений M и L. Функция шифрования в этом случае может быть отображена следующим образом:

    $$E = P(\rho^iR\rho^{-i})(\rho^{j}M\rho^{-j})(\rho^{k}L\rho^{-k})U(\rho^kL^{-1}\rho^{-k})(\rho^{j}M^{-1}\rho^{-j})(\rho^{i}R^{-1}\rho^{-i})P^{-1}$$

\subsubsection{Криптоанализ}
Попытки «взломать» Энигму не предавались гласности до конца 1970-х. После этого интерес к Энигме значительно возрос, и множество шифровальных машин представлено к публичному обозрению в музеях США и Европы.

В Немецком музее в Мюнхене находятся оба немецких военных варианта трёх- и четырёхроторной Энигмы, есть и устаревшие гражданские модели. Работающая модель представлена также в Международном шифровальном музее в Форт-Миде (Fort Meade), в Музее компьютерной истории (Computer History Museum) в США, в Блетчли-Парке (Bletchley Park) в Великобритании, в Австралийском военном мемориале (Australian War Memorial) в Канберре, а также в Германии, США, Великобритании и в некоторых других странах Европы.

В 2007 году запущен проект распределённых вычислений Enigma@Home, целью которого является взлом трех зашифрованных сообщений Энигмы, перехваченных в северной Атлантике в 1942 году.
