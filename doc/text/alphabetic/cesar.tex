\subsection{Шифр Цезаря}

Шифр Цезаря — это вид шифра подстановки, в котором каждый символ в открытом тексте заменяется буквой находящейся на некоторое постоянное число позиций левее или правее него в алфавите. Например, в шифре со сдвигом 3 А была бы заменена на Г, Б станет Д, и так далее.

Шифр назван в честь римского императора Гая Юлия Цезаря, использовавшего его для секретной переписки со своими генералами.

\subsubsection{Описание}

Если сопоставить каждому символу алфавита его порядковый номер (нумеруя с 0), то шифрование и дешифрование можно выразить формулами модульной арифметики:

    $$y=(x+k)\ \mod\ n$$
    $$x=(y-k+n)\ \mod\ n,$$

где $x$ — символ открытого текста, $y$ — символ шифрованного текста, $n$ — мощность алфавита, а $k$ — ключ.

\subsubsection{Криптоанализ}

Шифр Цезаря может быть легко взломан даже в случае, когда взломщик знает только зашифрованный текст. Можно рассмотреть две ситуации:

1. взломщик знает (или предполагает), что использовался простой шифр подстановки, но не знает, что это — схема Цезаря;

2. взломщик знает, что использовался шифр Цезаря, но не знает значение сдвига.

В первом случае шифр может быть взломан, используя те же самые методы что и для простого шифра подстановки, такие как частотный анализ и т. д., Используя эти методы, взломщик, вероятно, быстро заметит регулярность в решении и поймёт, что используемый шифр — это шифр Цезаря.

Во втором случае, взлом шифра является даже более простым. Существует не так много вариантов значений сдвига (26 для английского языка), все они могут быть проверены методом грубой силы. Один из способов сделать это — выписать отрывок зашифрованного текста в столбец всех возможных сдвигов — техника, иногда называемая как «завершение простого компонента». Рассмотрим пример для зашифрованного текста «EXXEGOEXSRGI»; открытый текст немедленно опознается глазом в четвертой строке.

Другой способ применения этого метода — это написать алфавит под каждой буквой зашифрованного текста, начиная с этой буквы. Метод может быть ускорен, если использовать заранее подготовленные полоски с алфавитом. Для этого нужно сложить полоски так, чтобы в одной строке образовался зашифрованый текст, тогда в некоторой другой строке мы увидим открытый текст.

Для обычного текста на естественном языке, скорее всего, будет только один вариант декодирования. Но, если использовать очень короткие сообщения, то возможны случаи, когда возможны несколько вариантов расшифровки с различными сдвигами. Например зашифрованный текст MPQY может быть расшифрован как «aden» так и как «know» (предполагая, что открытый текст написан на английском языке). Точно также «ALIIP» можно расшифровать как «dolls» или как «wheel»; «AFCCP» как «jolly» или как «cheer».

Многократное шифрование никак не улучшает стойкость, так как применение шифров со сдвигом a и b эквивалентно применению шифра со сдвигом a+b. В математических терминах шифрование с различными ключами образует группу.
