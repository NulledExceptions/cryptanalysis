\begin{abstract}
Изучение методов защиты неразрывно связано с изучением 
возможных атак на алгоритмы и на их реализации. 
Работы по анализу таких шифров, как DES, ГОСТ 28147-89, Blowfish требуют 
большого ресурса и являются чрезвычайно сложными. 
В то же время на примерах классических шифров можно 
проиллюстрировать некоторые важные приемы и методы криптоанализа. 
После анализа классических шифров возможно изучение современных 
блочных алгоритмов шифрования, становятся доступными идеи 
линейного и дифференциального криптоанализа.

В этой работе предпринята попытка написания фреймворка 
для криптоанализа классических шифров и оценки стойкости 
современных шифров.
\end{abstract}
