Во время написания курсовой работы мною были изучены классические
и блочные симметричные шифры на примере алгоритмов
шифр Цезаря, Афинный шифр, шифр Виженера, Xor, 
Enigma и Blowfish.

Передо мной стояла задача программной реализации программной 
реализации фреймворка для криптоанализа,
который имел бы удобный интерфейс пользователя и разработчика.

Эта задача была успешно выполнена. Фреймворк реализован, 
шифрует и расшифровывает файлы в соответствии с описаниями
изученных алгоритмов. В процессе написания мною были изучены некоторые 
тонкости программирования, связанные с шифрованием, а также с 
написанием программных интерфейсов.

Исходный код Python разработанного фреймворка и исходный 
код \TeX~курсовой работы доступны в Интернете в виде git 
репозитория по ссылке
\url{https://github.com/ch3sh1r/cryptanalysis}.
