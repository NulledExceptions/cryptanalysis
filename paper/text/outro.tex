\DEF\textit{Криптограмма} (шифротекст) --- результат операции шифрования.

\DEF\textit{Криптология} --- наука, занимающаяся методами шифрования и 
дешифрования. 

Криптология состоит из двух частей --- криптографии и криптоанализа. 
Криптография 
занимается разработкой методов шифрования данных, в то время как 
криптоанализ занимается оценкой сильных и слабых сторон методов 
шифрования, а также разработкой методов, позволяющих взламывать 
криптосистемы. Слово «криптология» (англ. cryptology) встречается в 
английском языке с XVII века, и изначально означало «скрытность в речи»; 
в современном значении было введено американским учёным Уильямом Фридманом 
и популяризовано писателем Дэвидом Каном \cite{khan-codebreakers}.

\DEF\textit{Криптоанализ} --- 
наука о методах расшифровки зашифрованной информации без предназначенного 
для такой расшифровки ключа.

В большинстве случаев под криптоанализом понимается выяснение ключа; 
криптоанализ включает также методы выявления уязвимости криптографических 
алгоритмов или протоколов. Первоначально методы криптоанализа основывались 
на лингвистических закономерностях естественного текста и реализовывались 
с использованием только карандаша и бумаги. Со временем в криптоанализе 
нарастает роль чисто математических методов, для реализации которых 
используются специализированные криптоаналитические компьютеры.
