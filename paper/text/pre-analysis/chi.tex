\subsection{Критерий $\chi^2$ (хи-квадрат)}

\DEF\textit{Критерий $\chi^2$ (хи-квадрат)}, или критерий Пирсона — наиболее часто употребляемый критерий для проверки гипотезы о законе распределения.

Для проверки критерия вводится статистика:

    $$\chi^2 = N\sum \frac{\left( P_i^{\mathrm{emp}} - P_i^{\mathrm{H_{0}}} \right)^2}{P_i^{\mathrm{H_{0}}}},$$

где $P_i^{\mathrm{H_{0}}} = F(x_i) - F(x_{i-1})$ — предполагаемая вероятность попадания в $i$-й интервал, $P_i^{\mathrm{emp}} = \frac{n_i}{N}$ — соответствующее эмпирическое значение, ${n_i}$ — число элементов выборки из $i$-го интервала, $N$ — полный объём выборки. Также используется расчет критерия по частоте, тогда:

    $$\chi^2 = \sum \frac{\left( V_i^{\mathrm{}} - NP_i^{\mathrm{H_{0}}} \right)^2}{NP_i^{\mathrm{H_{0}}}},$$

где $V_i$ — частота попадания значений в интервал. Эта величина, в свою очередь, является случайной (в силу случайности $\chi$) и должна подчиняться распределению $\chi^2$.
