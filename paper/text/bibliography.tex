\addcontentsline{toc}{section}{Список литературы}
\begin{thebibliography}{1}
\bibitem{shannon-inf} Шеннон К., «Работы по теории информации и кибернетике» (перевод Писаренко), 1963
\bibitem{fomi} Фомичев В.М., «Дискретная математика и криптология», 2003
\bibitem{yash} Ященко В.В., «Введение в криптографию», 1988
\bibitem{vaudenay-blowfish} Vaudenay S., «On the weak keys of Blowfish», 1996
\bibitem{khan-codebreakers} Khan D., The Codebreakers --- The Story of Secret Writing Revised edition (ISBN 978-0-684-83130-9) (1996)
\bibitem{gillogly-enigma} Gillogly J., «Ciphertext only Cryptanalysis of the Enigma», 1995
\bibitem{gillogly-enigma-corr} Erskine D., «Letter originally appeared in Cryptologia», 1996, \url{http://web.archive.org/web/20060720035430/http://members.fortunecity.com/jpeschel/erskin.htm}
\bibitem{google-ngrams} Franz A. and Brants T., «All Our N-gram are Belong to You», 2006, \url{http://googleresearch.blogspot.com.au/2006/08/all-our-n-gram-are-belong-to-you.html}
\bibitem{williams-enigma} Williams H., «Applying Statistical Language Recognition Techniques in the Ciphertext only Cryptanalysis of Enigma», 2005
\bibitem{segaran-data} Hammerbacher K. and Segaran M., «Beautiful Data», 2009
\bibitem{iso639} Стандарт представления наименований языков ISO 639-1:2002, \url{http://www.infoterm.info/standardization/iso_639_1_2002.php}
\end{thebibliography}
