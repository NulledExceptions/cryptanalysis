\DEF\textit{Криптограмма} (шифротекст) --- результат операции шифрования.

Объектом исследования является возможность чтения шифротекста
в условиях минимальной информированности о способе 
зашифрования. В часности, исследованы случаи неизвестного 
ключа при известном алгоритме шифрования и неизвестного 
алгоритма шифрования.

\DEF\textit{Криптология} --- наука, занимающаяся методами шифрования и 
дешифрования. 

Криптология состоит из двух частей --- криптографии и криптоанализа. 
Криптография 
занимается разработкой методов шифрования данных, в то время как 
криптоанализ занимается оценкой сильных и слабых сторон методов 
шифрования, а также разработкой методов, позволяющих взламывать 
криптосистемы. Слово «криптология» (англ. cryptology) встречается в 
английском языке с XVII века, и изначально означало «скрытность в речи»; 
в современном значении было введено американским учёным Уильямом Фридманом 
и популяризовано писателем Дэвидом Каном \cite{khan-codebreakers}.

\DEF\textit{Криптоанализ} --- 
наука о методах расшифровки зашифрованной информации без предназначенного 
для такой расшифровки ключа.

В большинстве случаев под криптоанализом понимается выяснение ключа; 
криптоанализ включает также методы выявления уязвимости криптографических 
алгоритмов или протоколов. Первоначально методы криптоанализа основывались 
на лингвистических закономерностях естественного текста и реализовывались 
с использованием только карандаша и бумаги. Со временем в криптоанализе 
нарастает роль чисто математических методов, для реализации которых 
используются специализированное криптоаналитическое программное 
обеспечение.

Для исследования сферы программных методов криптоанализа 
реализован фреймворк, в удобной форме объединяющий все 
необходимое для автоматизированного анализа классических 
алгоритмов шифрования.

\DEF\textit{Фреймворк} (англ. framework --- каркас, структура) ---
структура программной 
системы; программное обеспечение, облегчающее разработку и объединение 
разных компонентов большого программного проекта. 

Употребляется 
также слово «каркас», а некоторые авторы используют его в качестве 
основного, в том числе не базируясь вообще на англоязычном аналоге.
Можно также говорить о каркасном подходе как о 
подходе к построению программ, где любая конфигурация программы 
строится из двух частей: первая, постоянная часть — каркас, не 
меняющийся от конфигурации к конфигурации и несущий в себе гнезда
, в которых размещается вторая, переменная часть — сменные модули 
(или точки расширения) --- интерфейс пользователя, программный 
продукт.

Для программирования выбран язык Python 3.
Такой выбор обоснован ориентированностью Python на быстное 
портотипирование.
В программировании это значит возможность быстро написать 
структуру приложения и затем в плотную заниматься функциональностью
при необходимости переписывая узкие места алгоритма на 
компилируемых языках (рефакторинг).

Фреймворк использует только средства, идущие в стандартной 
поставке Python. Это позволяет использовать его сразу 
после установки интерпретатора. Основная функциональность 
фреймворка не завязана ни на архитектуру системы, ни 
на установленную операционную систему, что позволяет 
говорить о кросcплатформенности фреймворка. То есть, 
все работает в UNIX-подобных системах и запускается в 
Windows с незначительными потерями работоспособности.
